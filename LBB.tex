\chapter{最优约束比例}
本章用一种新的方式推导了LBB稳定性条件,建立了LBB稳定性条件与特征值问题和压力自由度之间的关系,提出了更精确的最优约束比。
\section{LBB稳定系数估计}
对于LBB稳定性条件\eqref{infsup},假设$\mathcal P_h:V_h \rightarrow Q_h$ 是$\mathcal P$的正交投影算子,定义为:
\begin{equation}\label{Ph}
    b(q_h,\boldsymbol v_h) = (q_h, \mathcal P \boldsymbol v_h) = (q_h, \mathcal P_h \boldsymbol v_h), \quad \forall q_h \in Q_h
\end{equation}
式中, $\mathcal P:V\rightarrow Q$ 散度算子, $\mathcal P = \nabla \cdot$. 

正如$\mathcal P_h$的定义,有$\mathrm{Im}\mathcal P_h \in Q_h$。式\eqref{infsup}可以改写为:
\begin{equation} \label{r11}
    \begin{split}
        \beta &\le \inf_{q_h \in Q_h} \sup_{\boldsymbol v_h \in V_h} \frac{\vert b(q_h,\boldsymbol v_h) \vert}{\Vert q_h \Vert_Q \Vert \boldsymbol v_h \Vert_V} 
        \le \inf_{q_h \in \mathrm{Im}\mathcal P_h} \sup_{\boldsymbol v_h \in V_h} \frac{\vert (q_h,\mathcal P_h \boldsymbol v_h) \vert}{\Vert q_h \Vert_Q \Vert \boldsymbol v_h \Vert_V} \\
    \end{split}
\end{equation}

对于给定的 $q_h\in \mathrm{Im}\mathcal P_h$,假设空间$V'_h \subset V_h\setminus \ker P_h$定义为:
\begin{equation}
    V'_h = \{ \boldsymbol v_h \in V_h \; \vert \; \mathcal P_h \boldsymbol v_h = q_h \}
\end{equation}
由 $\mathrm{Im}\mathcal P_h \in Q_h$,根据柯西--施瓦茨不等式,有
\begin{equation}
    \vert (q_h,\mathcal P_h \boldsymbol v_h) \vert \le \Vert q_h \Vert_Q \Vert \mathcal P_h \boldsymbol v_h \Vert_Q
\end{equation}
当且仅当$q_h=\mathcal P_h \boldsymbol v_h$时,等式成立,即
\begin{equation}
    \vert (q_h,\mathcal P_h \boldsymbol v_h) \vert = \Vert q_h \Vert_Q \Vert \mathcal P_h \boldsymbol v_h \Vert_Q, \quad \forall \boldsymbol v_h \in V'_h
\end{equation}
代入到式\eqref{r11}中可得:
\begin{equation}\label{r12}
    \sup_{\boldsymbol v_h\in V_h} \frac{\vert (q_h,\mathcal P_h \boldsymbol v_h) \vert}{\Vert q_h \Vert_Q \Vert \boldsymbol v_h \Vert_V} =
    \sup_{\boldsymbol v_h\in V'_h} \frac{\Vert \mathcal P_h \boldsymbol v_h \Vert_Q}{\Vert \boldsymbol v_h \Vert_V} 
\end{equation}

结合式\eqref{r11}、\eqref{r12}可得LBB稳定系数估算式:
\begin{equation}\label{r1}
    \beta \le \inf_{V'_h \subset V_h \setminus \ker \mathcal P_h} \sup_{v_h \in V'_h} \frac{\Vert \mathcal P_h \boldsymbol v_h \Vert_Q}{\Vert \boldsymbol v_h \Vert_V}
\end{equation}
其中 $\ker \mathcal P_h \subset V$ 是 $\mathcal P_h$ 的核,定义为 $\ker \mathcal P_h := \{ \boldsymbol v \in V \;\vert\; \mathcal P_h \boldsymbol v = 0 \}$.

上述方法与传统数值inf--sup测试一致\cite{malkus1981},而根据极小--极大值原理 \cite{babuska1991a},式\eqref{r1}计算刚度矩阵$\boldsymbol K^v$ 和 $\boldsymbol K^d$的最小非零特征值。

为了进一步找出最优约束数量,假设$P_{n_u}$ 是一个$n_u$ 维的多项式空间, $V_{n_u}$ 是位移多项式空间,且 $V_{n_u} = P_{n_u}^2$。由于$V_h$ 和 $V_{n_u}$的维度相同,$\dim V_{n_u}=\dim V_h = n_d\times n_u$,存在一个唯一的$\boldsymbol v \in V_{n_u}$ 满足$\boldsymbol v_h = \mathcal I_h \boldsymbol v$。于是式\eqref{r1}的右边可以改写为:
\begin{equation}\label{r21}
    \inf_{V'_h \subset V_h\setminus \ker \mathcal P_h}\sup_{\boldsymbol v_h \in V_h'} \frac{\Vert \mathcal P_h \boldsymbol v_h \Vert_Q}{\Vert \boldsymbol v_h \Vert_V} = 
    \inf_{V'\subset V_{n_u}\setminus \ker \mathcal P_h \mathcal I_h}\sup_{\boldsymbol v \in V'} \frac{\Vert \mathcal P_h \mathcal I_h \boldsymbol v \Vert_Q}{\Vert \mathcal I_h \boldsymbol v \Vert_V}
\end{equation}
压力的自由度为$n_p = \dim(V_{n_u}\setminus \ker \mathcal P_h \mathcal I_h)$

根据三角不等式、柯西--施瓦茨不等式和式\eqref{Ph}中的等式关系可得:
\begin{equation}\label{interpolation1}
    \begin{split}
        \Vert \mathcal P_h \mathcal I_h \boldsymbol v \Vert_Q &= 
        \sup_{q_h \in Q_h} \frac{\vert (q_h, \mathcal P_h \mathcal I_h \boldsymbol v) \vert}{\Vert q_h \Vert_Q}
        =\sup_{q_h \in Q_h} \frac{\vert (q_h, \mathcal P \mathcal I_h \boldsymbol v) \vert}{\Vert q_h \Vert_Q} \\
        &\le \sup_{q_h \in Q_h} \frac{\vert (q_h, \mathcal P \boldsymbol v)\vert + \vert (q_h, \mathcal P \boldsymbol v - \mathcal P \mathcal I_h \boldsymbol v) \vert}{\Vert q_h \Vert_Q} \\
        &= \Vert \mathcal P_h \boldsymbol v \Vert_Q
        + \Vert \mathcal P(\mathcal I - \mathcal I_h) \boldsymbol v \Vert_Q \\
    \end{split}
\end{equation}
显然,式\eqref{interpolation1}右侧的第二项和第三项是$V_h$中近似值的插值误差和正交投影误差,可以由如下式子估计\cite{yosida1995}:
\begin{equation}\label{interpolation2}
        \Vert \mathcal P(\mathcal I - \mathcal I_h) \boldsymbol v \Vert_Q \le Ch \vert \boldsymbol v \vert_{H} 
\end{equation}
从闭图像定理\cite{quarteroni1994}可得 $\Vert \mathcal I_h \boldsymbol v\Vert_V \ge C\Vert \boldsymbol v \Vert_V$。结合式\eqref{interpolation1}--\eqref{interpolation2},式\eqref{r21}的右边可以表示为:
\begin{equation}\label{r23}
    \begin{split}
        \inf_{V'\subset V_{n_u}\setminus \ker \mathcal P_h \mathcal I_h} \sup_{\boldsymbol v \in V'} \frac{\Vert \mathcal P_h\mathcal I_h\boldsymbol v\Vert_Q}{\Vert \mathcal I_h \boldsymbol v\Vert_V} 
        &\le \inf_{V'\subset V_{n_u}\setminus \ker \mathcal P_h \mathcal I_h} \sup_{\boldsymbol v \in V'} \frac{\Vert \mathcal P \boldsymbol v\Vert_Q}{\Vert \boldsymbol v\Vert_V} + Ch \\
    \end{split}
\end{equation}
将等式\eqref{r21},\eqref{r23}替换为\eqref{r1} 可以得到以下关系:
\begin{equation}\label{r3}
    \beta \le \beta_s + Ch
\end{equation}
其中
\begin{equation}
    \beta = \inf_{V'\subset V_{n_u}\setminus\ker \mathcal P_h \mathcal I_h}\sup_{\boldsymbol v \in V'}\frac{\Vert \mathcal P \boldsymbol v\Vert_Q}{\Vert  \boldsymbol v\Vert_V} 
\end{equation}

压力的最优自由度$n_p$为$n_s = \dim(V_{n_u}\setminus \ker \mathcal P)$
\section{多项式约束数}

根据上述内容,$n_c$为最优压力自由度,确定$n_c$的方法如下。例如,在二维弹性体积不可压缩问题中,对于维数为3的线性多项式空间$P_3$,对应的位移空间$V_3$由下式给出:
\begin{equation}
    V_3 = \mathrm{span} \left \{
    \begin{pmatrix} 1 \\ 0 \end{pmatrix},
    \begin{pmatrix} 0 \\ 1 \end{pmatrix},
    \begin{pmatrix} x \\ 0 \end{pmatrix},
    \begin{pmatrix} 0 \\ x \end{pmatrix},
    \begin{pmatrix} y \\ 0 \end{pmatrix},
    \begin{pmatrix} 0 \\ y \end{pmatrix}
    \right \}
\end{equation}
或者按照如下式子重新排列:
\begin{equation}\label{base1}
    V_3 = \mathrm{span} 
    \begin{Bmatrix}
        \underbrace{
        \begin{pmatrix} 1 \\ 0 \end{pmatrix},
        \begin{pmatrix} 0 \\ 1 \end{pmatrix},
        \begin{pmatrix} y \\ 0 \end{pmatrix},
        \begin{pmatrix} 0 \\ x \end{pmatrix},
        \begin{pmatrix} x \\ -y \end{pmatrix}
        }_{\ker \mathcal P},
        \underbrace{
        \begin{pmatrix} x \\ y \end{pmatrix}
        }_{V_3\setminus \ker \mathcal P}
    \end{Bmatrix}
\end{equation}
从式\eqref{base1}可以看出,$n_u=3$,$n_s=1$。根据这个方法,具有二次多项式基的位移空间$V_6$可以表述为:
\begin{equation}\label{base2}
    V_6 = \mathrm{span}
    \begin{Bmatrix}
        \overbrace{
        \begin{pmatrix} 1 \\ 0 \end{pmatrix},
        \begin{pmatrix} 0 \\ 1 \end{pmatrix},
        \begin{pmatrix} y \\ 0 \end{pmatrix},
        \begin{pmatrix} 0 \\ x \end{pmatrix},
        \begin{pmatrix} x \\ -y \end{pmatrix},
        \begin{pmatrix} x^2 \\ -2xy \end{pmatrix},
        \begin{pmatrix} y^2 \\ 0 \end{pmatrix},
        \begin{pmatrix} 0 \\ x^2 \end{pmatrix},
        \begin{pmatrix} -2xy \\ y^2 \end{pmatrix}
        }^{\ker \mathcal P}, \\
        \underbrace{
        \begin{pmatrix} x \\ y \end{pmatrix},
        \begin{pmatrix} x^2 \\ 2xy \end{pmatrix},
        \begin{pmatrix} 2xy \\ y^2 \end{pmatrix}
        }_{V_6\setminus \ker \mathcal P}
    \end{Bmatrix}
\end{equation}
在这种情况下,$n_u=6$,$n_s=3$。表中列出了随着多项式空间阶数的增加,各阶空间的最优约束自由度的数量,并总结了$n_u$与$n_s$之间的关系。
\begin{table}[ht!]
    \centering
    \caption{体积约束自由度}\label{tab:constraint}
    \setlength{\tabcolsep}{10mm}
    \renewcommand{\arraystretch}{2}
    \begin{tabular}{cccc}
        \toprule
            $n_u$ & $2n_u$ & $n$ &$ n_s$\\
        \midrule
        3  & 6  & 1 & 1 \\
        6  & 12 & 2 & 3 \\
        10 & 20 & 3 & 6 \\
        15 & 30 & 4 & 10 \\
        \vdots & \vdots & \vdots & \vdots \\
        $n_u$ & $2n_u$ & $\lfloor\frac{\sqrt{1+8n_u}-3}{2}\rfloor$ & $\frac{n(n+1)}{2}$  \\
        \bottomrule
    \end{tabular}
\end{table}
\section{有限元无网格混合离散}
第二章介绍的混合离散方案压力节点要依附单元上,无法任意布置。为了使压力自由度达到最优,在所提出的混合公式中,使用传统有限元法来近似位移,采用再生核无网格法来近似压力。
\subsection{再生核近似}
如图所示,再生核无网格近似将整个域$\Omega$及其边界$\Gamma$由$n_p$个无网格节点离散$\{\boldsymbol x_I\}_{I=1}^{n_p}$。每个无网格节点$\boldsymbol x_I$对应的形函数为$\Psi_I$,节点系数为$p_I$。于是近似压力$p_h$可以表示为:
\begin{equation}
    p_h(\boldsymbol x) = \sum_{I=1}^{n_p} \Psi_I(\boldsymbol x) p_I
\end{equation}
根据再生核近似理论,无网格形函数可以假设为如下形式:
\begin{equation}\label{rkshape}
    \Psi_I(\boldsymbol x) = \boldsymbol c(\boldsymbol x_I-\boldsymbol x) \boldsymbol p^{[n]}(\boldsymbol x_I-\boldsymbol x) \phi(\boldsymbol x_I - \boldsymbol x)
\end{equation}
其中$\boldsymbol p$为$n$阶基函数向量,其表达式为:
\begin{equation}
    \boldsymbol p^{[n]}(\boldsymbol x) = \{ 1, x, y, x^2, xy, y^2,...,y^n\}^T
\end{equation}
而$phi$为核函数,其影响域的大小由影响域尺寸$s$决定,核函数及其影响域的大小共同决定了无网格形函数的局部紧支性和光滑性。在二维情况下,核函数的影响域通常为圆形或者矩形。本文的影响域形状为矩形,矩形影响域的核函数可由下列公式计算得到:
\begin{equation}
    \phi(\boldsymbol x_I-\boldsymbol x) = \phi(r_x) \phi(r_y), \quad r_x = \frac{|\boldsymbol x_I - \boldsymbol x|}{s_{x}},r_y = \frac{|\boldsymbol y_I - \boldsymbol y|}{s_{y}}
\end{equation}
其中, $s_x$ 和 $s_y$ 分别为 $x$ 和 $y$ 方向上的影响域尺寸,若节点均匀布置,一般使两个方向上的影响域大小相等,即$s_x = s_y = s$。
为保证形函数紧支性和光滑性,$\phi$ 通常取为阶次大于 $n$ 的紧支函数。对于弹性力学问题,无网格基函数一般选择二阶或者三阶多项式基函数,核函数 $\phi(\boldsymbol x_I-\boldsymbol x)$ 取为三次样条函数:
\begin{equation}
    \phi(s) =\frac{1}{3!} \begin{cases}
        (2-2s)^3 - 4(1-2s)^3 & s\le\frac{1}{2} \\
        (2-2s)^3 &\frac{1}{2}<s<1 \\
        0 & s> 1
    \end{cases}
\end{equation}
$\boldsymbol c$为待定系数向量,可以通过满足下列一致性条件确定:
