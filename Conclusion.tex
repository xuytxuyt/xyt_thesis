\chapter{结论与展望}
\section{结论}
本文研究以衡量变量间自锁程度的约束比为基础,提出了一种自稳定的有限元无网格混合离散分析方法。
该方法创新的结合了再生核无网格数值积分方案、约束比与LBB稳定性条件,不仅为免自锁问题提供了一种可任意调节约束比的混合离散方法,还构建了一种内禀最优约束比的有限元无网格混合离散方案。该方案具有以下优势:满足LBB稳定性条件,确保了数值求解的高精度和可靠性;最优约束比范围,约束比始终处于最优范围内,有效缓解了自锁现象;节点布置简便,无需依赖单元几何拓扑关系,简化了节点布置的计算量;整体计算高效、稳定,消除了应力振荡现象。具体结论如下:

首先,本文以体积不可压材料的体积自锁问题为研究对象,构建考虑约束比的LBB稳定系数估计式,并确定了该问题的最优约束比范围。
在体积自锁问题的混合公式中,变量为位移和压力。当前缓解体积自锁的混合离散方案依赖于单元几何拓扑关系来布置节点,且约束比与LBB稳定性条件间的关系不明确,常常表现为过度约束的状态。
针对这一问题,本文从泛函分析的角度出发,验证了LBB稳定性系数与特征值问题的联系,并引入适定的完备多项式空间对LBB稳定系数估计式进行进一步化简,从而建立了考虑约束比的LBB稳定系数估计式。
结合体积不可压问题的完备多项式空间,得出了免体积自锁的最优约束比范围。
该估计式明确界定了满足LBB稳定性条件的自由度范围,相较于传统的数值验证特征值问题和解析证明框架,本文提出的估计式具有验证过程简单、直观且使用便利的显著优势。

其次,本文针对体积不可压问题,提出了一种有限元无网格混合离散分析方法,并确定了满足约束比和LBB稳定性条件的混合离散方案。
基于再生核无网格法的思想,位移采用传统有限元形函数进行离散,压力采用再生核无网格形函数进行离散。
与传统有限元形函数相比,无网格形函数具有高阶连续光滑的特点,不依赖于单元几何拓扑关系布置压力节点,从而可以任意调整约束比。
在有限元无网格混合离散框架下,本文深入研究位移、压力离散节点分布与LBB稳定性条件之间的关系,验证LBB稳定系数估计式的正确性。
通过对LBB稳定系数进行系统性测试,本文提出了免体积自锁的有限元无网格混合离散方案。
该方案的压力节点直接基于位移节点选取,严格满足LBB稳定性条件,兼具简易性与稳定性。
为验证所提方案的有效性,本文采用了一系列体积不可压问题的经典算例,测试了其计算精度和稳定性,并以传统最优约束比下的混合离散方案作为对比项。
数值结果表明,本文所提方案不仅能够保证计算精度和理论误差收敛率,而且在压力稳定性方面表现优异,彻底消除了压力振荡现象,其稳定性要优于传统混合离散方案。

最后,本文将有限元无网格混合离散分析方法进一步推广至中厚板剪切自锁问题。
在剪切自锁问题中,混合公式的变量为挠度和剪切应力。同样的将挠度采用传统有限元形函数进行离散,剪切应力采用无网格形函数进行离散。
基于LBB误差估计式和中厚板问题的完备多项式空间,推导出了中厚板免剪切自锁的最优约束比范围。
与体积不可压问题相比,中厚板问题最优的挠度自由度与剪切应力自由度之间的关系更为简单。
为验证所提方法的有效性,采用中厚板经典算例进行数值实验。
结果表明,基于LBB误差估计式得出的最优约束比范围与数值结果高度吻合。
在此基础上,本文提出了一种免剪切自锁的有限元无网格混合离散方案,并通过算例验证了其计算精度和稳定性。
与传统最优约束比下的离散方案相比,本文所提方案在保证计算精度和理论误差收敛率的同时,能够有效消除应力振荡现象,进一步证明了其在解决中厚板剪切自锁问题中的优越性。

本文所提方法可以同时满足最优约束比和LBB稳定性条件,避免自锁现象,可以为精确分析自锁问题提供一种稳定、可靠和高效的数值仿真工具。

\section{展望}
自稳定有限元无网格混合离散分析方法可以同时满足最优约束比与LBB稳定性条件,计算稳定,节点布置便利,且不会产生应力振荡现象。本文仅用经典算例对本方法进行验证,离真正的应用还有差距。后续针对本文所提方法的研究工作可以针对以下几点开展:

(1)发展针对壳体结构的有限元无网格混合分析方法。壳体结构同样存在着剪切自锁与薄膜自锁问题,所提方法结合相应的混合公式即可推广至壳体结构。

(2)本文仅考虑线弹性的材料模型,后续研究可将所提有限元无网格混合离散分析方法推广至材料非线性和大变形情况,以将该方法应用到结构损伤破坏分析中。

(3)将所提方法推广至实际应用。如橡胶密封件的压缩和变形问题、金属塑性成形(如锻造、冲压、挤压)过程、生物软组织(如肌肉、血管、皮肤)模拟等体积不可压问题。