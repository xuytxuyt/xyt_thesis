\chapter{体积自锁问题}
本章以体积不可压材料的本构特性,说明了传统有限元分析时产生的体积自锁现象的原因。针对体积自锁问题,系统介绍了当前主流的两种解决方案:一是基于罚函数法的缩减积分方案,二是基于拉格朗日乘子法的混合离散方案,并通过等效投影验证二者的等价性。在此基础上,结合缓解体积自锁的典型单元,详细阐述并验证免体积自锁的两个关键条件,包括体积约束比和LBB稳定性条件。

\section{体积不可压材料}               
考虑如图\ref{model}所示维度为$n_d$且具有边界$\Gamma$的体积不可压材料弹性体$\Omega\in \mathbb R^{n_d}$,$\Gamma_t$和$\Gamma_g$分别表示其自然边界和本质边界,
并满足$\Gamma_t \cup \Gamma_g = \Gamma$,$\Gamma_t \cap \Gamma_g = \varnothing$。该问题相应的强形式为:
\begin{equation}\label{strong_penalty}
    \begin{cases}
        \nabla \cdot \boldsymbol \sigma + \boldsymbol b = \boldsymbol 0 & \mathrm{in} \; \Omega \\
        \boldsymbol \sigma \cdot \boldsymbol n = \boldsymbol t & \mathrm{on} \; \Gamma_t \\
        \boldsymbol u = \boldsymbol g & \mathrm{on} \; \Gamma_g \\
\end{cases}
\end{equation}
其中$\boldsymbol \sigma$为应力张量,对于各向同性线弹性材料,其本构关系表示为:
\begin{equation}\label{stress_penalty}
    \boldsymbol \sigma(\boldsymbol u) = 3\kappa \boldsymbol \varepsilon^v(\boldsymbol u) + 2\mu \boldsymbol \varepsilon^d(\boldsymbol u) 
\end{equation}
式中$\boldsymbol \varepsilon^v$ 和 $\boldsymbol \varepsilon^d$ 为应变张量$\boldsymbol \varepsilon$的体积应变和偏应变部分:
\begin{equation}
    \boldsymbol \varepsilon^v(\boldsymbol u) =\frac{1}{3} \nabla \cdot \boldsymbol u \; \boldsymbol 1, \quad
    \boldsymbol \varepsilon^d(\boldsymbol u) =\frac{1}{2}(\boldsymbol u \nabla + \nabla \boldsymbol u) - \boldsymbol \varepsilon^v, \quad
    \boldsymbol \varepsilon^v : \boldsymbol \varepsilon^d = 0
\end{equation}
其中,$\boldsymbol 1 = \delta_{ij} \boldsymbol e_i \otimes \boldsymbol e_j$是二阶恒等张量。$\kappa$, $\mu$ 分别为体积模量和剪切模量,其与杨氏模量$E$和泊松比$\nu$之间存在如下关系式:
\begin{equation}\label{modulus}
    \kappa = \frac{E}{3(1-2\nu)}, \quad \mu = \frac{E}{2(1+\nu)}
\end{equation}
$\boldsymbol b$为$\Omega$中的体力,$\boldsymbol t$,$\boldsymbol g$ 分别为自然边界$\Gamma_t$和本质边界$\Gamma_g$上的牵引力和位移。
\begin{figure}[!h]
    \centering 
        \includegraphics[scale=0.1]{figures/model.png}
        \caption{不可压缩材料弹性体模型}\label{model}
\end{figure}

对于体积不可压缩材料,其泊松比$\nu \rightarrow 0.5$。 在这种情况下,式\eqref{modulus}中体积模量$\kappa \rightarrow \infty$,而剪切模量$\mu$的变化相对较小,从而导致$\kappa\gg\mu$。
根据体积不可压缩材料本构关系\eqref{stress_penalty},当$\kappa \rightarrow \infty$时,体积应变$\boldsymbol \varepsilon^v$被约束,导致体积的变化$\nabla \cdot \boldsymbol u\rightarrow 0$。

采用伽辽金法进行求解时,其相对应的伽辽金弱形式为:
\begin{equation}\label{weak_penalty}
    \begin{split}
        \text{Find}&\,\boldsymbol u \in V\\
        &\int_\Omega 2\mu \delta \boldsymbol \varepsilon^d : \boldsymbol \varepsilon^d d\Omega +
        \int_\Omega 3\kappa \delta \boldsymbol \varepsilon^v : \boldsymbol \varepsilon^v d\Omega =
        \int_{\Gamma_t} \delta \boldsymbol u \cdot \boldsymbol t d\Gamma + \int_\Omega \delta \boldsymbol          u \cdot \boldsymbol b d\Omega, \quad
        \forall \delta \boldsymbol u \in V
    \end{split}
\end{equation}
其中,空间$V=\{\boldsymbol v \in H^1(\Omega)^2\;\vert\;\boldsymbol v = \boldsymbol g, \; \textrm{on} \; \Gamma_g\}$。
$\delta\boldsymbol \varepsilon^v$和$\delta\boldsymbol \varepsilon^d $分别为$\delta \boldsymbol u$表示的体积应变和偏应变的变分。

在传统有限元法中,如图\ref{fem}所示整个求解域$\Omega$可离散为一组节点$\{\boldsymbol x_I\}_{I=1}^{n_u}$表示\cite{hughes2000},其中$n_u$是位移节点的数量。
位移$\boldsymbol u$及其变分$\delta \boldsymbol u $可通过$\boldsymbol x_I$处的节点系数和形函数进行近似:
\begin{equation}\label{u_h}
    \boldsymbol u_h(\boldsymbol x) = \sum_{I=1}^{n_u} N_I(\boldsymbol x) \boldsymbol u_I, \quad
    \delta \boldsymbol u_h(\boldsymbol x) = \sum_{I=1}^{n_u} N_I(\boldsymbol x) \delta \boldsymbol u_I
\end{equation}
其中$N_I$和$\boldsymbol u_I$分别为节点$\boldsymbol{x}_I$处的形函数和节点系数张量。
\begin{figure}[H]
    \centering 
        \includegraphics[scale=0.05]{figures/fem.png}
        \caption{有限元离散示意图}\label{fem}
\end{figure}

将式\eqref{u_h}代入到弱形式\eqref{weak_penalty}中可得下列的里兹--伽辽金问题:
\begin{equation}\label{ritz_penalty}
    \begin{split}
        \text{Find}&\, \boldsymbol u_h \in V_h\\
        &\int_\Omega 2\mu \delta \boldsymbol \varepsilon^d_h : \boldsymbol \varepsilon^d_h d\Omega +
        \int_\Omega 3\kappa \delta \boldsymbol \varepsilon^v_h : \boldsymbol \varepsilon^v_h d\Omega =
        \int_{\Gamma_t} \delta \boldsymbol u_h \cdot \boldsymbol t d\Gamma + \int_\Omega \delta         \boldsymbol u_h \cdot \boldsymbol b d\Omega, \quad
        \forall \delta \boldsymbol u_h \in V_h
    \end{split}
\end{equation}
其中近似空间$V_h \subseteq V$,$V_h = \{\boldsymbol v_h \in (\mathrm{span}\{N_I\}_{I=1}^{n_u})^{n_d} \vert \boldsymbol v_h = \boldsymbol g,\; \mathrm{on} \; \Gamma_g\}$。
根据$\delta \boldsymbol u_h$的任意性,等式两边同时消除$\delta \boldsymbol u_I$,上述方程可以简化为如下离散控制方程:
\begin{equation}\label{equilibrium_penalty}
    (\boldsymbol K^d +\boldsymbol K^v) \boldsymbol d^u = \boldsymbol f
\end{equation}
其中,$\boldsymbol d^u$ 是包含 $\boldsymbol u_I$的系数向量,$\boldsymbol K^v$ 为主应力刚度矩阵,$\boldsymbol K^d$为偏应力刚度矩阵,$\boldsymbol f$为力向量,其分量分别为:
\begin{equation}\label{stiffness_vol}
    \boldsymbol K^v_{IJ}=  3\kappa\int_{\Omega} \boldsymbol B^{v\mathrm T}_I \boldsymbol B^v_J d\Omega
\end{equation}
\begin{equation}\label{stiffness_dev}
    \boldsymbol K^d_{IJ}= 2\mu\int_{\Omega} \boldsymbol B^{d\mathrm T}_I \boldsymbol B^d_J d\Omega
\end{equation}
\begin{equation}
    \boldsymbol f_I = \int_{\Gamma_t} N_I \boldsymbol t d\Gamma + \int_{\Omega} N_I \boldsymbol b d\Omega
\end{equation}
式中$\boldsymbol B^v_I$和$\boldsymbol B^d_I$为形函数梯度矩阵,在二维问题中具有如下表达式:
\begin{equation}
    \boldsymbol{B}^v_I= \left[\begin{matrix}
        \frac{1}{3}N_{I,x}&\frac{1}{3}N_{I,y}\\
        \frac{1}{3}N_{I,x}&\frac{1}{3}N_{I,y}\\
        0&0\\
        \frac{1}{3}N_{I,x}&\frac{1}{3}N_{I,y}
    \end{matrix}\right] ,\quad
    \boldsymbol{B}^d_I= \left[\begin{matrix}
        \frac{2}{3}N_{I,x}&-\frac{1}{3}N_{I,y}\\
        -\frac{1}{3}N_{I,x}&\frac{2}{3}N_{I,y}\\
        \frac{1}{2}N_{I,y}&-\frac{1}{2}N_{I,x}\\
        -\frac{1}{3}N_{I,x}&-\frac{1}{3}N_{I,y}
    \end{matrix}\right]
\end{equation}

\section{免体积自锁方案}
\subsection{罚函数法与缩减积分方案}

由式\eqref{equilibrium_penalty},\eqref{stiffness_vol}可知,对于体积不可压材料,当外力向量$\boldsymbol f$具有一定的数值时,$\kappa \rightarrow \infty$将导致主应力刚度矩阵$\boldsymbol K^v$中的所有模态也趋向零。
此时,主应力刚度矩阵$\boldsymbol K^v$可作为罚函数项将主应力刚度矩阵中所有模态约束住,体积模量$\kappa$为其中的罚因子。
使用传统有限元法进行求解时,由于离散的有限元近似阶次较低,导致过多的位移自由度受到体积约束的限制,位移解将远小于实际情况,即出现体积自锁现象。

引入数值积分实现主应力刚度矩阵\eqref{stiffness_vol},有:
\begin{equation}\label{integration}
    \boldsymbol K^v_{IJ} =  3\kappa \sum_{C=1}^{n_e}\sum_{G=1}^{n_g} \boldsymbol B^{vT}_I(\boldsymbol x_G) \boldsymbol B^v_J(\boldsymbol x_G) w_G
\end{equation}
式中,$\boldsymbol x_G$和$w_G$分别为积分点的位置和权重。$n_e$是单元的数量,$n_g$是每个单元中积分点的数量。当引入数值积分后,$\boldsymbol K^v$将要求体积约束在所有数值积分点处满足,$\mathrm{rank}(\boldsymbol K^v)$将受限于数值积分点数量:
\begin{equation}\label{rank}
    \mathrm{rank}(\boldsymbol K^v)\le n_e \times n_g
\end{equation}

由式\eqref{rank}可知,减少用于主应力刚度矩阵的积分点数量可以减少体积约束的位移自由度个数。
以图\ref{reduced}所示的经典二维四边形单元(Quad4)缩减积分方案为例,Quad4单元需要采用$2\times2$高斯积分点以保证数值精度,称之为完全积分方案。
为了缓解体积自锁现象,Quad4单元在对主应力刚度矩阵$\boldsymbol K^v$进行数值积分时,采用1点高斯积分作为缩减积分方案。
此时,被约束的自由度个数将被单元数$n_e$限制:
\begin{equation}
    \mathrm{rank}(\boldsymbol K^v)\le n_e
\end{equation}
\begin{figure}[!h]
    \centering
        \begin{tabular}{c@{\hspace{24pt}}c}
            \includegraphics[width=0.3\textwidth]{figures/rd_quad4.png} &
            \includegraphics[width=0.3\textwidth]{figures/rd_q4r1.png} \\
            完全积分 & 缩减积分 \\
            \raisebox{-0.3\height}{\includegraphics[width=14pt]{figures/legend_u.png}} :位移节点 &
            \raisebox{-0.3\height}{\includegraphics[width=14pt]{figures/legend_i.png}} :高斯积分点 
        \end{tabular}
        \caption{Quad4单元缩减积分方案}\label{reduced}
\end{figure}

需要指出在有限元离散中,单元数量$n_e$将远小于位移自由度个数$2n_u$。
例如单边单元数为$n$的正方形均布有限元离散,此时的单元数量为$n_e = n^2$,而位移自由度个数为$2n_u = 2(n+1)^2 = 2n^2+4n+2$,$n_e < 2n_u$,体积自锁现象得以缓解。

相反,为保证整体刚度矩阵的正定性,对偏应力刚度矩阵$\boldsymbol{K^d}$采用完全积分方案进行数值积分。此时,$\boldsymbol{K^d}$被约束的位移自由度个数被节点数$n_u$限制:
\begin{equation}
    \mathrm{rank}(\boldsymbol K^d) \le 2n_u
\end{equation}

由上述例子可知,缩减积分方案减少数值积分时用于主应力刚度矩阵高斯积分点的数量,从而减少了体积约束位移自由度的个数,缓解了体积自锁现象。

\subsection{拉格朗日乘子法与混合离散方案}
施加约束的方式除了传统罚函数法外,常用的方法还有拉格朗日乘子法。
对于体积自锁问题,将压力$p$作为拉格朗日乘子独立变量,可以得到拉格朗日乘子型体积约束施加方案,相对应的强形式中增加了压力$p$与体积应变$\nabla \cdot \boldsymbol u$之间的约束:
\begin{equation}\label{strong_mix}
    \begin{cases}
        \nabla \cdot \boldsymbol \sigma + \boldsymbol b = \boldsymbol 0 & \mathrm{in} \; \Omega \\
        \frac{p}{\kappa} + \nabla \cdot \boldsymbol u = 0 & \mathrm{in} \; \Omega \\
        \boldsymbol \sigma \cdot \boldsymbol n = \boldsymbol t & \mathrm{on} \; \Gamma_t \\
        \boldsymbol u = \boldsymbol g & \mathrm{on} \; \Gamma_g \\
    \end{cases}
\end{equation}
式中应力张量$\boldsymbol \sigma$采用两个变量表示:
\begin{equation}\label{stress_mix}
    \boldsymbol \sigma(\boldsymbol u, p) = p \boldsymbol 1 + 2\mu \boldsymbol \varepsilon^d(\boldsymbol u)
\end{equation}
其中$p\in Q$,$Q = \{q \in L^2(\Omega) \vert \int_{\Omega} q d\Omega = 0\}$。此时,能量泛函具有位移$\boldsymbol u$和压力$p$两个变量,分别对两变量进行变分,可得相对应的伽辽金弱形式为:
\begin{flalign}
    &\text{Find}\, \boldsymbol u \in V, p \in Q&\nonumber
\end{flalign}
\begin{equation}\label{ch_2:eq:weak_mix}
    \begin{aligned}
        a(\delta \boldsymbol u, \boldsymbol u) + b(\delta \boldsymbol u, p) &= f(\delta \boldsymbol u) \quad &\forall \delta \boldsymbol u \in V \\
        b(\boldsymbol u, \delta p) +  c(\delta p,p)&= \boldsymbol 0 \quad &\forall \delta p \in Q
    \end{aligned}
\end{equation}
式中,$a: V\times V\rightarrow \mathbb R$,$b: V\times Q\rightarrow \mathbb R$,$c: V\times V\rightarrow \mathbb R$为双线性算子, $f: V \rightarrow \mathbb R$ 为线性算子,它们具有以下形式:
\begin{align}
    a(\delta \boldsymbol u, \boldsymbol u) &= \int_\Omega \nabla^s \delta \boldsymbol u: \nabla^s \boldsymbol u d\Omega \\
    b(\delta \boldsymbol u, p) &= \int_\Omega \nabla \cdot \delta \boldsymbol u p d\Omega \\
    c(\delta p,p) &= -\frac{1}{\kappa}\int_\Omega\delta p p d\Omega \\
    f(\delta \boldsymbol u) &= \int_{\Gamma_t} \delta \boldsymbol u \cdot \boldsymbol t d\Gamma + \int_{\Omega} \delta \boldsymbol u \cdot \boldsymbol b d\Omega
\end{align}

采用伽辽金法进行求解时,位移$\boldsymbol u$和压力$p$双变量可以采用不同的离散节点进行近似,形成混合离散框架。
近似的位移$u_h$和压力$p_h$及其变分可表示为:
\begin{equation}\label{ch_2:eq:u_h_mix}
    \boldsymbol u_h(\boldsymbol x) = \sum_{I=1}^{n_u} N_I(\boldsymbol x) \boldsymbol u_I, \quad
    \delta \boldsymbol u_h(\boldsymbol x) = \sum_{I=1}^{n_u} N_I(\boldsymbol x) \delta \boldsymbol u_I
\end{equation}
\begin{equation}\label{ch_2:eq:p_h_mix}
    p_h(\boldsymbol x) = \sum_{K=1}^{n_p} \Psi_K(\boldsymbol x) p_K, \quad
    \delta p_h(\boldsymbol x) = \sum_{K=1}^{n_p} \Psi_K(\boldsymbol x) \delta p_K
\end{equation}
式中,$n_p$分别为压力节点的总数,$p_K$为压力节点系数,$\Psi_K$为离散$p_h$的形函数。

根据$\delta \boldsymbol  u_h$和$\delta p_h$的任意性,式\eqref{ch_2:eq:weak_mix}可得到如下离散控制方程:
\begin{equation}\label{equilibrium_mix}
    \begin{bmatrix}
        \boldsymbol K^{uu} & \boldsymbol K^{up} \\ (\boldsymbol K^{up})^{\mathrm T} & \boldsymbol K^{pp}
    \end{bmatrix}
    \begin{Bmatrix}
        \boldsymbol d^u \\ \boldsymbol d^p 
    \end{Bmatrix} =
    \begin{Bmatrix}
        \boldsymbol f \\ \boldsymbol 0 
    \end{Bmatrix}
\end{equation}
式中,$\boldsymbol K^{uu} = \boldsymbol K^d$, $\boldsymbol{K}^{up}$为刚度矩阵中和位移、压力都有关的部分,$\boldsymbol{K}^{pp}$为刚度矩阵中只和压力相关的部分,其分量分别为:
\begin{equation}
    \boldsymbol{K}^{up}_{I J}=\int_{\Omega} \boldsymbol B_{I}^{v\mathrm T} \Psi_{J} {d} \Omega
\end{equation}
\begin{equation}  
    \boldsymbol{K}^{pp}_{IJ}=-\frac{1}{\kappa}\int_{\Omega}  \Psi^\mathrm{T}_{I}  \Psi_{J}d\Omega
\end{equation}

为统计约束的自由度个数,将离散控制方程式\eqref{equilibrium_mix}进行变换。
由式\eqref{equilibrium_mix}可将$\boldsymbol d^p$采用$\boldsymbol d^u$表示:
\begin{equation}
    \boldsymbol d^p = (\boldsymbol K^{pp})^{-1} (\boldsymbol K^{up})^{\mathrm T} \boldsymbol d^u
\end{equation}

将上式代入到式\eqref{equilibrium_mix}的第一式中可得:
\begin{equation}\label{ch_2:eq:equilibrium_projection}
    \begin{split}
        &(\underbrace{\boldsymbol K^{uu}}_{\boldsymbol K^d} +  \underbrace{\boldsymbol K^{up}(\boldsymbol K^{pp})^{-1}(\boldsymbol K^{up})^{\mathrm T}}_{\tilde{\boldsymbol K}^v}) \boldsymbol d^u = \boldsymbol f \\
        \Rightarrow\;& (\boldsymbol K^d + \tilde{\boldsymbol K}^v) \boldsymbol d^u= \boldsymbol f
    \end{split}
\end{equation}

控制方程\eqref{ch_2:eq:equilibrium_projection}与式\eqref{equilibrium_penalty}具有相同的形式。为保证积分精度,混合离散方案对主应力刚度矩阵$\tilde{\boldsymbol K}^v$和偏应力刚度矩阵$\boldsymbol{K^d}$进行数值积分时均采用完全积分,而采用不同的节点个数进行离散。因此,体积约束的自由度个数被节点数限制。以图\ref{mix_ex}所示的常用于缓解体积自锁的Q4P1单元混合离散方案为例,为缓解体积自锁现象,Q4P1单元采用4个位移节点离散$\boldsymbol{K^d}$;采用1个压力节点离散$\tilde{\boldsymbol K}^v$。此时,被约束的自由度个数为:
\begin{equation}
    \mathrm{rank}(\boldsymbol K^d)\le 2n_u
\end{equation}
\begin{equation}
    \mathrm{rank}(\tilde{\boldsymbol K}^v)=\min(\mathrm{rank}({\boldsymbol K}^{up}),\mathrm{rank}({\boldsymbol K}^{pp}))\le n_p
\end{equation}

需要指出的是在上述混合离散方案中,压力自由度的个数始终小于位移自由度的个数。例如单边单元数为$n$的正方形均布Q4P1单元混合离散,此时压力自由度个数为$n_p=n^2$,而位移自由度个数为$2n_u=2(n+1)^2=2n^2+4n+2$,$n_p < 2n_u$,缓解体积自锁现象。
\begin{figure}[!h]
    \centering
        \begin{tabular}{c@{\hspace{24pt}}c}
            \multicolumn{2}{c}{\includegraphics[width=0.3\textwidth]{figures/mix_Q4P1.png}}\\
            \raisebox{-0.3\height}{\includegraphics[width=14pt]{figures/legend_u.png}} :位移节点 &
            \raisebox{-0.3\height}{\includegraphics[width=14pt]{figures/legend_p.png}} :压力节点 
        \end{tabular}
    \caption{Q4P1单元混合离散方案}\label{mix_ex}
\end{figure}

由上述例子可知,混合离散方案对位移$\boldsymbol{u}$和压力$p$两个变量采用不同数量的位移节点和压力节点进行离散,使压力自由度个数少于位移自由度个数,缓解体积自锁现象。
\subsection{罚函数法和拉格朗日乘子法方法的等价性}
由上述小节可知,罚函数型和拉格朗日乘子型缓解体积自锁方案均改变了体积约束自由度的个数,缓解了体积自锁现象。从Quad4单元和Q4P1单元两个例子可以看出,$n_g$与$n_p$具有类似的效果。同样,也可通过数值积分法的等效投影证明两类方法的等价性。

从拉格朗日乘子法的离散方程\eqref{ch_2:eq:equilibrium_projection}第二式中可以看出,压力$p_h$的解是$3\kappa \nabla \cdot \boldsymbol u_h$的一个正交投影。
令$\mathcal P_h$为正交投影算子,满足:
\begin{equation}\label{ch_2:eq:orthogonal}
    (q_h,\mathcal P_h \boldsymbol u_h) = (q_h, \nabla \cdot \boldsymbol u_h), \quad \forall q_h \in Q_h
\end{equation}
其中,$(\bullet,\bullet)$为内积算子。
为表示方便,可将投影后的位移散度写作$p_h=\mathcal P_h \boldsymbol u_h = \tilde \nabla \cdot \boldsymbol u_h$。且有$\tilde \nabla \cdot \boldsymbol u_h \in \mathrm{Im} \mathcal P_h$,$\mathrm{Im} \mathcal P_h \in Q_h$为$\mathcal P_h$的相空间\cite{philippeg.2013}。
此时,式\eqref{ch_2:eq:orthogonal}可改写为:
\begin{equation}
    \int_\Omega q_h(\nabla \cdot \boldsymbol u_h - \tilde \nabla \cdot \boldsymbol u_h) d\Omega = 0, \quad \forall q_h \in Q_h
\end{equation}
将上式代入弱形式\eqref{equilibrium_mix}中,其主应力部分可转化为:
\begin{equation}\label{projection_mixed}
    \begin{split}
        \int_\Omega \nabla \cdot \delta \boldsymbol u_h p_h d\Omega &= \underbrace{\int_\Omega (\nabla \cdot \delta  \boldsymbol u_h - \tilde \nabla \cdot \delta \boldsymbol u_h) p_h d\Omega}_0 + \int_\Omega \tilde \nabla \cdot \delta \boldsymbol u_h \underbrace{p_h}_{\tilde \nabla \cdot \boldsymbol u_h} d\Omega \\
        &= \int_\Omega 3\kappa \tilde \nabla \cdot \delta \boldsymbol u_h \tilde \nabla \cdot \boldsymbol u_h d\Omega \\
    \end{split}
\end{equation}
里兹--伽辽金变分方程\eqref{ritz_penalty}变为:
\begin{equation}\label{ch_2:eq:ritz_mix}
    \begin{split}
        &\text{Find} \,\boldsymbol u_h \in V_h\\
    &\int_\Omega 2\mu \delta \boldsymbol \varepsilon^d_h : \boldsymbol \varepsilon^d_h d\Omega +
    \int_\Omega 3\kappa \tilde \nabla \cdot \delta \boldsymbol u_h \tilde \nabla \cdot \boldsymbol u_h d\Omega =
    \int_{\Gamma_t} \delta \boldsymbol u_h \cdot \boldsymbol t d\Gamma + \int_\Omega \delta \boldsymbol u_h \cdot \boldsymbol b d\Omega, \quad \forall \boldsymbol u_h \in V_h
    \end{split}
\end{equation}

与此同时对于罚函数法,数值积分也可以被视作一种投影。设$\varrho_i$为正交多项式满足:
\begin{equation}
    \int_{\Omega_C} \varrho_i \varrho_j d\Omega = 
    \begin{cases}
        w_i  & i = j \\
        0 & i \ne j
    \end{cases}
\end{equation}
正交插值 $\mathcal T^{k}: V_h \rightarrow W^{k}$,其中 $W^{k}$ 是由$k$个正交多项式构成的插值空间:
\begin{equation}
    W^{k}:= \mathrm{span}\{\varrho_i \}_{i=1}^{k}
\end{equation}

对于传统高斯积分方案,$\varrho_i(\boldsymbol x_j) = \delta_{ij}$, $\boldsymbol x_j$是积分点的位置。体积应变可以通过正交插值法表示为:
\begin{equation}
    \nabla \cdot \boldsymbol u_h(\boldsymbol x) \approx \bar \nabla \cdot \boldsymbol u_h(\boldsymbol x) = \sum_{G=1}^{n_g} \varrho_G(\boldsymbol x) \nabla \cdot \boldsymbol u_h(\boldsymbol x_G), \quad \nabla \cdot \boldsymbol u_h(\boldsymbol x_G) = \bar \nabla \cdot \boldsymbol u_h(\boldsymbol x_G)
\end{equation}
而积分点被视为插值系数。积分点的总数$n_g$低于完全积分,这意味着 $\nabla \cdot \boldsymbol u_h$ 投影到子空间。弱形式\eqref{ritz_penalty}中,主应力部分可转化为:
\begin{equation}\label{projection_penalty}
    \begin{split}
        \int_\Omega 3\kappa \delta \boldsymbol \varepsilon^v_h : \boldsymbol \varepsilon^v_h d\Omega &=\int_\Omega 3\kappa \nabla \cdot \delta \boldsymbol u_h \nabla \cdot \boldsymbol u_h d\Omega  \\
        &= \sum_{C=1}^{n_e} \sum_{G=L=1}^{n_g} 3\kappa \nabla \cdot \delta \boldsymbol u_h(\boldsymbol x_G) \nabla \cdot \boldsymbol u_h(\boldsymbol x_G) w_G \\
        &= \sum_{C=1}^{n_e} \sum_{G,L=1}^{n_g} 3\kappa \nabla \cdot \delta \boldsymbol u_h(\boldsymbol x_G) \nabla \cdot \boldsymbol u_h(\boldsymbol x_L) \int_\Omega \varrho_G \varrho_L d\Omega  \\
        &=\int_\Omega 3\kappa \bar\nabla \cdot \delta \boldsymbol u_h \bar\nabla \cdot \boldsymbol u_h d\Omega
    \end{split}
\end{equation}
里兹--伽辽金变分方程\eqref{ritz_penalty}可改写为:
\begin{equation}\label{ritz_penalty_new}
    \begin{split}
    &\text{Find}\,\boldsymbol u_h \in V_h\\
    &\int_\Omega 2\mu \delta \boldsymbol \varepsilon^d_h : \boldsymbol \varepsilon^d_h d\Omega +
    \int_\Omega 3\kappa \bar \nabla \cdot \delta \boldsymbol u_h \bar \nabla \cdot \boldsymbol u_h d\Omega =
    \int_{\Gamma_t} \delta \boldsymbol u_h \cdot \boldsymbol t d\Gamma + \int_\Omega \delta \boldsymbol u_h \cdot \boldsymbol b d\Omega, \quad \forall \boldsymbol u_h \in V_h
    \end{split}
\end{equation}

通过对比式\eqref{ritz_penalty_new}和式\eqref{ch_2:eq:ritz_mix},罚函数法实际上与拉格朗日乘子法等价,两种方法都可以用投影的方式来描述,两种方法具有等价性。

\section{免体积自锁条件}
\subsection{体积约束比}
从上文介绍的两种常用的缓解体积自锁方法可知,压力自由度的数量在缓解体积自锁中有关键的作用。文献{\citestyle{numbers}\cite{hughes2000}}提出了约束比的概念。约束比是用来衡量变量间的自锁程度的重要指标。对于体积不可压问题而言,约束比定义为位移总自由度与压力总自由度的比值。文献{\citestyle{numbers}\cite{hughes2000}}还通过连续控制方程确定了最优的约束比数量。由体积不可压问题连续控制方程\eqref{strong_mix}可知,在第一式当中位移在全域上的约束为$n_d$个方程,而第二式当中位移与压力的关系为$2:1$。在离散形式下也应满足上述关系。
\begin{equation}
    r = \frac{n_d\times n_u}{n_p}, \quad 
    \begin{cases}
        r > n_d & \text{过少的约束} \\
        r = n_d & \text{最优} \\
        r < n_d & \text{过多的约束} \\
    \end{cases}
\end{equation}

以图\ref{pressure_elements}所示的两个不同的二维单元混合离散方案为例,图中阴影部分为边界条件。
图\ref{pressure_elements}(a)为线性位移--常数压力三角形单元(T3P1),有4个位移节点和2两个压力节点。对应的总的位移自由度个数为$4\times2=8$,而3个位移节点受到边界条件约束,因此位移自由度个数为$8-3\times2=2$。压力自由度个数为$2\times1=2$。此时,该单元的约束比为:
\begin{equation}
    r=\frac{2}{2}=1
\end{equation}
表明此单元压力约束自由度过多,不能缓解体积自锁现象,与实际单元的情况一致。

图\ref{pressure_elements}(b)为双线性位移--常数压力四边形单元(Q4P1),有4个位移节点和1个压力节点。相应的总的位移自由度个数为8,且3个位移节点受到边界条件约束,位移自由度的个数为$2$。压力自由度个数为1。该单元的约束比为:
\begin{equation}
    r=\frac{2}{1}=2
\end{equation}
结果显示此单元满足最优约束比,能够缓解体积自锁。
\begin{figure}[!h]
    \centering
        \begin{tabular}{c@{\hspace{24pt}}c}
            \includegraphics[width=0.3\textwidth]{figures/T3P1.png} &
            \includegraphics[width=0.3\textwidth]{figures/Q4P1.png} \\
            (a) & (b) \\
            \raisebox{-0.3\height}{\includegraphics[width=14pt]{figures/legend_u.png}} :位移节点 &
            \raisebox{-0.3\height}{\includegraphics[width=14pt]{figures/legend_p.png}} :压力节点 
        \end{tabular}
        \caption{二维位移压力单元示例}\label{pressure_elements}
    \end{figure}

值得注意的是,最优约束比只是一种启发式手段,并不能确保单元组合能够缓解自锁现象。它主要作为一种便捷的工具来评估单元是否有缓解自锁的潜在能力,对于具有相同数量位移和压力节点的经典单元,尽管其约束比为最优,但在实际应用中仍会出现体积自锁问题。在全面评估单元性能时,还需要综合考虑其它相关因素。

\subsection{LBB稳定性条件}
进行免体积自锁单元性能评估时,还应考虑Ladyzhenskaya--Babuska--Brezzi(LBB)稳定性条件,也被称为inf--sup条件\cite{babuska1997a,bathe1996},是对免自锁方法更为精确的要求。这一条件基于混合离散框架构建,当满足inf--sup条件时,可以确保混合方程的准确性和稳定性。
\begin{equation}\label{ch:2:eq:LBB}
    \inf_{q_h \in Q_h} \sup_{\boldsymbol v_h \in V_h} \frac{\vert b(q_h,\boldsymbol v_h) \vert}{\Vert q_h \Vert_Q \Vert \boldsymbol v_h \Vert_V} \ge \beta > 0
\end{equation}
其中$\beta$为LBB稳定系数,是与单元尺寸$h$无关的常数。

然而,验证LBB稳定性条件比较困难,目前验证LBB稳定性条件的方法有两种。一种是数值验证,通过建立特征值问题\cite{chapelle1993}在数值分析中验证离散方案是否满足LBB稳定性条件;另一种是理论分析,确定离散方案是否被包含在始终满足LBB稳定性条件的解析证明框架\cite{chapelle1993}中验证其是否满足LBB稳定性条件。

在理论分析方面,以下解析证明框架始终满足LBB稳定性条件。通过确定离散方案是否包含在该框架内,可以验证其是否满足LBB稳定性条件。
\begin{equation}\label{analy}
    \begin{cases}
        &\Vert\Pi_1 w\Vert_V \le c_1\Vert w\Vert_W \\
        &b(\Pi_2\nu-\nu,q_h)=0 \quad \forall q_h \in Q_h\\
        &\Vert \Pi_2(I-\Pi_1)w \Vert_V \le c \Vert w\Vert_W \\
    \end{cases}
\end{equation}

建立特征值问题可以数值验证离散方案是否通过满足LBB稳定性条件。对于体积不可压问题,由式\eqref{ch_2:eq:equilibrium_projection}可得,主应力刚度矩阵 $\boldsymbol K^v$和偏应力刚度矩阵$ \boldsymbol K^d$有下列特征值关系。
\begin{equation}\label{ch_2:eq:eigenvalue}
    \boldsymbol K^d=\lambda\boldsymbol K^v,\quad \beta=\sqrt{\lambda_p}
\end{equation}
式中$\lambda_p$为最小非零特征值。通过改变单元尺寸$h$,验证$\lambda_p$是否与$h$无关来验证其是否满足LBB稳定性条件。

表\ref{infsuptest}详细列出了用于缓解体积自锁现象的经典单元,包括它们的约束比、数值验证以及解析证明结果。
从表中可以看出,传统混合有限元法中使用的单元在约束比最优的情况下无法满足LBB稳定性条件,而满足LBB稳定性条件的单元通常存在压力约束自由度不足的情况。这两种情况都会对结果的准确性和稳定性产生影响。

值得注意的是,尽管体积约束比条件简单直观,但其结果不准确。
相比之下,LBB稳定性条件虽然能够精确的验证结果且具有完备的理论,但其验证过程较为复杂,且缺乏直观性。
更为关键的是,体积约束比与LBB稳定性条件之间的内在关联尚未得到明确阐释,这也制约着体积不可压问题的有效解决。
\begin{table}[!h]
    \centering
    \renewcommand\arraystretch{1.2}
    \caption{LBB稳定性条件验证} \label{infsuptest}
    \begin{tabular}{cccc}
        \hline
        \multirow{2}{*}{离散方案}&体积&\multicolumn{2}{c}{LBB稳定性条件}\\
        & 约束比&数值验证&解析证明\\
        \hline
        \begin{tabular}{c}
            \begin{minipage}{0.13\columnwidth}
                \centering
                \includegraphics[width=0.9\textwidth]{figures/mix_T3P1.png}
            \end{minipage}\\T3P1
        \end{tabular}
        &1&$\times$ & $\times$\\
        \begin{tabular}{c}
            \begin{minipage}{0.13\columnwidth}
                \centering
                \includegraphics[width=0.9\textwidth]{figures/mix_Q4P1.png}
            \end{minipage}\\Q4P1
        \end{tabular}
        &2&$\times$ & $\times$\\
        \begin{tabular}{c}
            \begin{minipage}{0.13\columnwidth}
                \centering
                \includegraphics[width=0.9\textwidth]{figures/mix_Q8P3.png}
            \end{minipage}\\Q8P3
        \end{tabular}
        &2&$\times$ & $\times$\\
        \begin{tabular}{c}
            \begin{minipage}{0.13\columnwidth}
                \centering
                \includegraphics[width=0.9\textwidth]{figures/mix_Q8P1.png}
            \end{minipage}\\Q8P1
        \end{tabular}
        &6&\checkmark & \checkmark\\
        \begin{tabular}{c}
            \begin{minipage}{0.13\columnwidth}
                \centering
                \includegraphics[width=0.9\textwidth]{figures/mix_Q9P3.png}
            \end{minipage}\\Q9P3
        \end{tabular}
        &$\frac{8}{3}$&\checkmark & \checkmark\\
        \begin{tabular}{c}
            \begin{minipage}{0.13\columnwidth}
                \centering
                \includegraphics[width=0.9\textwidth]{figures/mini.png}
            \end{minipage}\\MINI element \cite{arnold1984,auricchio2005}
        \end{tabular}
        &$\frac{8}{3}$&\checkmark & \checkmark\\
        \begin{tabular}{c}
            \begin{minipage}{0.13\columnwidth}
                \centering
                \includegraphics[width=0.9\textwidth]{figures/TaylorHood.png}
            \end{minipage}\\Taylor--Hood element \cite{hood1974}
        \end{tabular}
        &$\frac{8}{3}$&\checkmark & \checkmark\\
        \begin{tabular}{c}
            \begin{minipage}{0.13\columnwidth}
                \centering
                \includegraphics[width=0.9\textwidth]{figures/mix_T6C3.png}
            \end{minipage}\\T6C3
        \end{tabular}
        &8&\checkmark & \\
        \begin{tabular}{c}
            \begin{minipage}{0.13\columnwidth}
                \centering
                \includegraphics[width=0.9\textwidth]{figures/CrouzeixRaviart.png}
            \end{minipage}\\Crouzeix-Raviart element \cite{crouzeix1973}
        \end{tabular}
        &4&\checkmark & \\
        \hline
        \multicolumn{4}{c}{
                \begin{tabular}{c@{\hspace{24pt}}c}
                    \raisebox{-0.2\height}{\includegraphics[width=10pt]{figures/legend_u.png}} :位移节点 &
                    \raisebox{-0.2\height}{\includegraphics[width=10pt]{figures/legend_p.png}} :压力节点 
                \end{tabular}}\\
        \hline
    \end{tabular}
\end{table}
\section{小结}
本章系统探讨了体积自锁问题及其解决方案。首先,从体积不可压材料的特性出发,结合离散控制方程表达式,分析了传统有限元法在处理体积不可压问题时产生体积自锁现象的根本原因。
随后,重点介绍了两种常用的体积自锁缓解方法:罚函数与缩减积分方案和拉格朗日乘子法与混合离散方案。
罚函数与缩减积分方案通过减少主应力刚度矩阵数值积分点的数量来缓解体积自锁现象;拉格朗日乘子法与混合离散方案则通过引入压力作为独立变量,并采用不同数量的位移和压力节点进行离散,以减少压力节点的数量来缓解体积自锁现象。
此外,通过数值积分法等效投影,证明了两种方案的等价性。最后,基于经典的混合离散方案,阐述了免体积自锁的两个关键条件——体积约束比和LBB稳定性条件,并详细介绍了如何通过连续控制方程确定最优体积约束比以及混合离散单元的约束比个数。同时,列举了经典混合离散方案的约束比,并结合两种LBB稳定性条件的验证方法,验证其是否满足LBB稳定性条件。
